% getestet für
%
% Firmware: 0.5~testing5
% Gerät: TP-Link TL-WR841N

\documentclass{article}

\usepackage[utf8]{inputenc}	%Zeichencodierung Text
\usepackage[T1]{fontenc}
\usepackage{textcomp}
\usepackage[ngerman]{babel}	%Wortdefinitionen
\usepackage[a5paper,%
 left=20mm,%
 right=20mm,%
 top=20mm,%
 bottom=20mm]{geometry}
\usepackage{fancyhdr}
 
\pagestyle{fancy}
\fancyhf{}
\rhead{bremen.freifunk.net}
\lhead{Freifunk Bremen}
\rfoot{Mehr Informationen zur Einrichtung unter wiki.bremen.freifunk.net}

\renewcommand{\thefootnote}{\roman{footnote}}

\begin{document}

\section*{Kurzanleitung zur Inbetriebnahme}

\begin{enumerate}
\item Schließe den Freifunk-Router mit dem mitgelieferten Netzteil an das Stromnetz an.

\item Stecke das mitgelieferte LAN-Kabel mit dem einen Ende in die \textbf{LAN}-Buchse deines Computers und mit dem anderen in eine \textbf{LAN}-Buchse des Freifunk-Routers.

\item Rufe im Browser des Computers \textbf{192.168.1.1} auf. \label{webinterface}

\item Konfiguriere deinen Freifunk-Router. 
\begin{enumerate}
  \item Setze ein Häkchen bei "`\textbf{Mesh-VPN aktivieren}"' um den Freifunk-Router über das Internet mit dem Freifunknetz zu verbinden. \label{vpn-mesh-aktivieren} \footnote{Einen Zugang zum Internet kann der Freifunk-Router nur bereitstellen, wenn dieser mit dem Freifunk-Netz verbunden ist. Mit dem Freifunk-Netz kann der Router per Mesh-VPN (erfordert eigenen Internetzugang) oder Mesh (erfordert Freifunk-Knoten in Reichweite des Geräts) verbunden werden.}
  
  \item Optional können Geokoordinaten\footnote{Dessen Ermittlung kannst du z.B. mit Hilfe des Teilen-Werkzeuges von osm.org vornehmen.} für den Standort des Freifunk-Routers und eine E-Mail-Adresse angegeben werden.

  \item Klicke unten rechts auf "Fertig". 

  \textit{Die Konfigurationsseite des Freifunk-Routers ist anschließend nicht mehr erreichbar.} \footnote{Durch 3-5 sekündiges Drücken der Reset-Taste wird der Konfigurationsmodus des Routers wieder aktiviert. Anschließend kann wie in Schritt \ref{webinterface} fortgefahren werden.}
\end{enumerate}

\item Entferne nun das LAN-Kabel.

\item Wenn für "`Mesh-VPN aktivieren"' in Schritt \ref{vpn-mesh-aktivieren} kein Haken gesetzt wurde, fahre mit Schritt \ref{platzieren} fort. 

\item Stecke das LAN-Kabel mit dem einen Ende in die \textbf{WAN}-Buchse deines Freifunk-Routers und mit dem anderen in eine \textbf{LAN}-Buchse deines Routers, der Verbindung zum Internet hat.\footnote{Der Freifunk-Router verbindet sich so per VPN (über das Internet) mit dem übrigen Freifunk Bremen Intranet.}

\item Platziere den Freifunk-Router an einem Ort deiner Wahl. \label{platzieren}

\item Fertig!
\end{enumerate}

\end{document}