% getestet für
%
% Firmware: 0.5~testing5
% Gerät: TP-Link TL-WR841N

\documentclass{article}

\usepackage[utf8]{inputenc}	%Zeichencodierung Text
\usepackage[T1]{fontenc}
\usepackage{textcomp}
\usepackage[ngerman]{babel}	%Wortdefinitionen
\usepackage[a5paper,%
 left=15mm,%
 right=15mm,%
 top=20mm,%
 bottom=20mm]{geometry}
\usepackage{fancyhdr}
 
\pagestyle{fancy}
\fancyhf{}
\rhead{bremen.freifunk.net}
\lhead{Freifunk Bremen}
\rfoot{Mehr Informationen zur Einrichtung unter wiki.bremen.freifunk.net}

\renewcommand{\thefootnote}{\roman{footnote}}

\begin{document}

\section*{Kurzanleitung zur Inbetriebnahme}

\begin{enumerate}
\item Schließe den Freifunk-Router mit dem mitgelieferten Netzteil an das Stromnetz an.

%
% in neuer Version müsste nach Julians Änderungen alles über den WAN-Port konfigurierbar sein

\item Stecke das mitgelieferte LAN-Kabel mit dem einen Ende in die \textbf{LAN}-Buchse deines Computers und mit dem anderen in eine \textbf{LAN}-Buchse des Freifunk-Routers.

\item Rufe im Browser des Computers \textbf{192.168.1.1} auf. \label{webinterface}

\item Konfiguriere deinen Freifunk-Router. 
\begin{enumerate}

%
% folgendes item sollte mit Julians Version default werden und dürfte daher entfallen
  \item Setze ein Häkchen bei \glqq{}\textbf{Mesh-VPN aktivieren}\grqq{} um den Freifunk-Router über das Internet mit dem Freifunknetz zu verbinden. \label{vpn-mesh-aktivieren} \footnote{Einen Zugang zum Internet kann der Freifunk-Router nur bereitstellen, wenn dieser mit dem Freifunk-Netz verbunden ist. Mit dem Freifunk-Netz kann der Router per Mesh-VPN (erfordert eigenen Internetzugang) oder Mesh (erfordert Freifunk-Knoten in Reichweite des Gerätes) verbunden werden.}
  
  \item Optional können Geokoordinaten\footnote{Mit Hilfe des Werkzeugs \textit{Koordinaten beim nächsten Klicken anzeigen} der Knotenkarte auf bremen.freifunk.net ermittelbar.} für den Standort des Freifunk-Routers und eine E-Mail-Adresse angegeben werden.

  \item Klicke unten rechts auf \glqq{}Fertig\grqq{} und lasse die Konfigurationsseite geöffnet. \label{fertig}

  \textit{Die Konfigurationsseite des Freifunk-Routers ist nun nicht mehr erreichbar.} \footnote{Durch 3-5 sekündiges Drücken der Reset-Taste wird der Konfigurationsmodus des Routers wieder aktiviert. Anschließend kann wie in Schritt \ref{webinterface} fortgefahren werden.}
\end{enumerate}

\item Entferne nun das LAN-Kabel.


%
% in neuer Version von Julian: Entferne nun das LAN-Kabel vom Computer und stecke es ggf. in eine LAN-Buchse deines Routers. -footnote-

\item Falls für \glqq{}Mesh-VPN aktivieren\grqq{} in Schritt \ref{vpn-mesh-aktivieren} ein Haken gesetzt wurde, stecke das LAN-Kabel mit dem einen Ende in die \textbf{WAN}-Buchse deines Freifunk-Routers und mit dem anderen in eine \textbf{LAN}-Buchse deines Routers, der Verbindung zum Internet hat.\footnote{Der Freifunk-Router verbindet sich so per VPN (Virtual Private Network) über das Internet mit dem übrigen Freifunk Bremen Intranet.}

\item Platziere den Freifunk-Router an einem Ort deiner Wahl. \label{platzieren}

%
% folgender Item entfällt (sofern ich das nicht falsch verstehe), wenn der Public-Key von fastd bereits beim Flashen an den Server übermittelt worden ist.

\item Folge den Instruktionen der Seite aus Schritt \ref{fertig}, sobald der Computer wieder eine Verbindung zum Internet hat.

\item Fertig!
\end{enumerate}

\end{document}